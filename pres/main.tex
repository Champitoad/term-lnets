\documentclass[usenames,dvipsnames]{beamer}
\usetheme{CambridgeUS}
\usepackage[utf8]{inputenc}
\usepackage{amsmath}
\usepackage{amssymb}
\usepackage{amsthm}
\usepackage{unicode-math}
\usepackage{mathtools}
\usepackage{mathpartir,proof}
\usepackage{prftree}
\usepackage{cmll}
\usepackage{tabu}
\usepackage{color}
\usepackage{xcolor}
\usepackage{pn}
\usepackage{makecell}
\usepackage{tikz-cd}
\usepackage{pbox}
\usepackage{adjustbox}

% \includeonlyframes{current}

% \setbeameroption{show notes}

\graphicspath{ {figures/} }
\useinnertheme{rectangles}
\setbeamertemplate{blocks}[default]

% More vertical spacing in tabu lines
\tabulinesep=3mm

\prflineextra=0em
\prflinepadbefore=0.5ex
\prflinepadafter=0.5ex
\prfrulenameskip=0.5em

\newcolumntype{L}[1]{>{\raggedright\let\newline\\\arraybackslash\hspace{0pt}}m{#1}}
\newcolumntype{C}[1]{>{\centering\let\newline\\\arraybackslash\hspace{0pt}}m{#1}}
\newcolumntype{R}[1]{>{\raggedleft\let\newline\\\arraybackslash\hspace{0pt}}m{#1}}

\newcommand{\lto}{\Rightarrow}
\newcommand{\lequiv}{\Leftrightarrow}
\newcommand{\lolli}{\multimap}
\newcommand{\dai}{✠}
\newcommand{\seq}{\vdash}
\newcommand{\irule}[1]{\footnotesize$#1$}
\newcommand{\iruleL}[1]{\irule{{#1}\seq}}
\newcommand{\iruleR}[1]{\irule{\seq{#1}}}
\newcommand{\cutbar}{\mathbin{\|}}
\newcommand{\cutred}{\rightsquigarrow}

\AtBeginDocument{% to do this after unicode-math has done its work
  \renewcommand{\setminus}{\mathbin{\backslash}}
}
\renewcommand{\b}[1]{\overline{#1}}

\title[Travail de Recherche Encadré]{Towards a term syntax for L-nets}
% \subtitle{Jean-Yves Girard}
\author{Pablo Donato}
\institute[]{Université Paris Diderot}
\date{\today}
\subject{Logic and Computer Science}

\setcounter{tocdepth}{1}
% \AtBeginSection[]
% {
%     \begin{frame}
%         \frametitle{Table of contents}
%         \tableofcontents[currentsection]
%     \end{frame}
% }

\begin{document}

\begin{frame}
    \titlepage
    \vspace{-2em}
    \center
    {\small Encadré par Alexis Saurin}
\end{frame}

\section*{Introduction}

\begin{frame}
    \frametitle{Curry-Howard isomorphism}
    \begin{center}
        \begin{tabu}{>{\bf\color{red}}c@{$\qquad \Longleftrightarrow \qquad$}>{\bf\color{blue}}c}
            proof & program \\
            formula & type \\
            normalization/cut elimination & evaluation/execution
        \end{tabu}
    \end{center}
    \begin{itemize}
        \setlength\itemsep{1em}
        % \item Relates logical \textcolor{red}{\textbf{proofs}} with (functional)
        % \textcolor{blue}{\textbf{programs}}
        % \item The proven \textcolor{red}{\textbf{formula}} corresponds to the program's
        % \textcolor{blue}{\textbf{type}}
        % \item Proof \textbf{\textcolor{red}{normalization}} (aka {\bf \color{red} cut elimination}) corresponds to program \textbf{\textcolor{blue}{execution}}
        \item Initially discovered for \textbf{\textcolor{red}{minimal intuitionistic}} natural
        deduction and \textbf{\textcolor{blue}{simply-typed $\lambda$-calculus}} (Howard, 1969)
        \item Extended to \textbf{\textcolor{red}{2nd-order intuitionistic}} natural deduction and
        \textbf{\textcolor{blue}{polymorphic $\lambda$-calculus/System F}} (Girard-Reynold, 1972)
    \end{itemize}
    % \begin{itemize}
    %     \setlength\itemsep{1em}
    %     \item Discovered in the 80s by J.-Y. Girard
    %     \item Usually presented as a "logic of resources"
    % \end{itemize}
\end{frame}

\begin{frame}[fragile]
    \frametitle{Motivations}
    \begin{center}
    \adjustbox{height=0.45\textheight,keepaspectratio}{
        \begin{tikzcd}[row sep=2em, column sep=2em, math mode=false]
            & \parbox{2.5cm}{\centering System F\\\scriptsize (Girard, 1972)} \arrow[d] & & \\
            \parbox{2.5cm}{\centering \bf Proof nets\\\scriptsize (Girard, 1987)} \arrow[ddr] \arrow[ddddrr, Rightarrow, bend right] & \parbox{2.5cm}{\centering \bf Linear Logic\\\scriptsize (Girard, 1987)} \arrow[l, Rightarrow]\arrow[dr] \arrow[rrddd, Rightarrow, bend left] & & \\
            & & \parbox{2.5cm}{\centering Ludics\\\scriptsize (Girard, 2001)} \arrow[dl]\arrow[ddr] & \\
            & \parbox{2.5cm}{\centering L-nets\\\scriptsize(FMC, 2005)} \arrow[ddr] & & \\
            & & & \parbox{2.5cm}{\centering \bf c-designs\\\scriptsize (Terui, 2011)} \arrow[dl, Rightarrow] \\
            & & \bf This work! & \\
        \end{tikzcd}
        % \begin{tikzcd}[row sep=2em, column sep=2em, math mode=false]
        %     System F \arrow[r] & 
        %     \parbox{3cm}{\centering {\bf\color{red} Linear Logic\\{\scriptsize (Girard, 1987)}}} \arrow[r, Rightarrow, red]\arrow[ddr] &
        % \end{tikzcd}
    }
    \end{center}
    % \begin{itemize}
    %     \setlength\itemsep{1.5em}
    %     \item 2nd order intuitionistic logic $\leftrightarrow$ polymorphic
    %     $\lambda$-calculus
    %     \item Disjunction property for $\oplus$, involutive negation $\cdot^\bot$
    %     \item Unordered proofs
    %     \item Program vs. environment
    % \end{itemize}
\end{frame}

% \begin{frame}
%     \frametitle{L-nets}
%     \fontsize{11pt}{7.2}\selectfont
%     \vspace{1em}
%     \begin{itemize}
%         \setlength\itemsep{2em}
%         \item A model of \textbf{interactive}, \textbf{concurrent} computation, introduced in 2005 by C.
%         Faggian and F. Maurel.
%         \item L-nets can be seen as \textbf{untyped}, \textbf{partially sequential} proofs of linear
%         logic
%         \item Our goal will be to introduce a \textbf{term syntax} equivalent to the graph
%         theoretical formulation of L-nets
%     \end{itemize}
%     % \begin{block}{Proof nets}
%     %     \begin{itemize}
%     %         \item A \textbf{parallel} syntax for the proofs of linear logic
%     %         \item Graph-theoretic formulation that forgets some "irrelevant" order of inferences
%     %     \end{itemize}
%     % \end{block}
%     % \begin{block}{Designs}
%     %     \begin{itemize}
%     %         \item The main objects of \textbf{ludics}, a foundational theory of logic
%     %         \item \textbf{Abstract} proofs where formulae are replaced by their \emph{address}
%     %         \item \textbf{Sequential} and \textbf{interactive} by nature
%     %     \end{itemize}
%     % \end{block}
%     % \vspace{3mm}
%     % \center
%     % L-nets = abstract proof nets = parallel designs
% \end{frame}

\section{Multiplicative linear logic}

\begin{frame}[fragile, label=current]
    \frametitle{Multiplicative linear logic}
    \color{black!25}
    \begin{center}
    \adjustbox{height=0.45\textheight,keepaspectratio}{
        \begin{tikzcd}[row sep=2em, column sep=2em, math mode=false]
            & \parbox{2.5cm}{\centering System F\\\scriptsize (Girard, 1972)} \arrow[d] & & \\
            \color{black!0} \parbox{2.5cm}{\centering \bf \color{black} Proof nets\\\scriptsize (Girard, 1987)} \arrow[ddr] \arrow[ddddrr, Rightarrow, bend right] & \color{black!0} \parbox{2.5cm}{\centering \bf \color{black} Linear Logic\\\scriptsize (Girard, 1987)} \arrow[l, Rightarrow, black]\arrow[dr] \arrow[rrddd, Rightarrow, bend left] & & \\
            & & \parbox{2.5cm}{\centering Ludics\\\scriptsize (Girard, 2001)} \arrow[dl]\arrow[ddr] & \\
            & \parbox{2.5cm}{\centering L-nets\\\scriptsize(FMC, 2005)} \arrow[ddr] & & \\
            & & & \parbox{2.5cm}{\centering \bf c-designs\\\scriptsize (Terui, 2011)} \arrow[dl, Rightarrow] \\
            & & \bf This work! & \\
        \end{tikzcd}
    }
    \end{center}
\end{frame}

\begin{frame}
    \frametitle{Linear logic}
    \small
    \begin{align*}
        A ~::=~&{\color{red} X} \mid {\color{red} X^{\bot}} \mid &\quad \text{(Atoms)}\\
        &\textbf{1} \mid \bot \mid {\color{red} A \otimes A} \mid {\color{red} A \parr A} \mid &\quad \text{\color{red} (Multiplicatives)}\\
        &\textbf{0} \mid \top \mid A \oplus A \mid A \with A \mid &\quad \text{(Additives)}\\
        &\oc A \mid \wn A &\quad \text{(Exponentials)}\\
    \end{align*}
    \vspace{-3em}
    \begin{itemize}
        \setlength\itemsep{0.75em}
        \item Fine management of formulae as \textbf{resources}
        \item Can encode both {\bf\color{violet} classical} and {\bf\color{orange} intuitionistic} logic
        \item Exhibits features of both worlds:
        \begin{itemize}
            \item {\bf\color{orange} Disjunction property} for $\oplus$
            \item {\bf\color{violet} Involutive negation} $\cdot^\bot$:
            \large
            $$A = A^{\bot\bot}$$
        \end{itemize}
    \end{itemize}
\end{frame}

\subsection{A focalized sequent calculus for $\mathsf{MLL}$}

\begin{frame}
    \frametitle{Sequents}
    \begin{itemize}
        \setlength\itemsep{1em}
        \item Formulae are generated by the following \textbf{polarized} grammar:
        \begin{align*}
            P &::= X ∣ N \otimes N ∣ \shpos N &\quad \text{\bf(positive)} \\
            N &::= X^\bot ∣ P \parr P ∣ \shneg P &\quad \text{\bf(negative)}
        \end{align*}
        \item Negation defined by \textbf{De Morgan} equations on dual connectives:
        \begin{align*}
            (X)^\bot &= X^\bot & (N_1 \otimes N_2)^\bot &= N_1^\bot \parr N_2^\bot & (\shpos N)^\bot &= \shneg N^\bot \\
            (X^\bot)^\bot &= X & (P_1 \parr P_2)^\bot &= P_1^\bot \otimes P_2^\bot & (\shneg P)^\bot &= \shpos P^\bot
        \end{align*}
        \item A \textbf{focalized sequent} is a multiset of formulae of the form:
        $$\seq A, P_1, \ldots, P_n \quad \text{also written} \quad \seq A, Γ$$
        % \item A \textbf{sequent} is simply a list of formulae, written:
        % $$\seq A_1, \ldots, A_n \quad \text{or} \quad \seq Γ$$
    \end{itemize}
\end{frame}

\begin{frame}
    \frametitle{Proofs}
    \begin{itemize}
        \setlength\itemsep{1em}
        \item A \textbf{proof} of a sequent is just a \textbf{derivation tree}
        % $$\prftree{\prfsummary{\seq \ldots}}{\ldots}{\prfsummary{\seq \ldots}}{\seq A, Γ}$$
        \item Leaves are \textbf{axiom} rules:
        \vspace{1em}
        $$\prfbyaxiom{\irule{\mathrm{ax}}}{\seq P^\bot, P}$$
        \item Nodes are either \textbf{cut} rules or \textbf{logical} rules:
        \vspace{1em}
        $$
        \prftree[r]{\irule{\mathrm{cut}}}
            {\seq P^\bot, Γ}
            {\seq P, Δ}
            {\seq Γ, Δ}
        $$
    \end{itemize}
\end{frame}

\begin{frame}
    \frametitle{Logical rules}
    They define how to prove \textbf{compound} formulae:
    \vspace{2em}
    \begin{mathpar}
        \prftree[r]{\irule{\otimes}}
            {\seq N_1, Γ}
            {\seq N_2, Δ}
            {\seq N_1 \otimes N_2, Γ, Δ}
        \and
        \prftree[r]{\irule{\parr}}
            {\seq P_1, P_2, Γ}
            {\seq P_1 \parr P_2, Γ}
        \\
        \prftree[r]{\irule{\shpos}}
            {\seq N, Γ}
            {\seq \shpos N, Γ}
        \and
        \prftree[r]{\irule{\shneg}}
            {\seq P, Γ}
            {\seq \shneg P, Γ}
    \end{mathpar}
\end{frame}

\subsection{Cut elimination}

\begin{frame}
    \frametitle{Cut elimination}
    \begin{itemize}
        \setlength\itemsep{1em}
        \item A \textbf{rewriting} procedure that \emph{eliminates} cut rules from proofs
        \item Through Curry-Howard, can be seen as an \textbf{evaluation strategy}
        \item Defined as the iteration of \textbf{cut reduction} steps:
        \vspace{1em}
        $$
        \vcenter{
        \prftree[r]{\irule{\mathrm{cut}}}
            {\prfbyaxiom{\irule{\mathrm{ax}}}{\seq P^\bot, P}}
            {\prfsummary[$π$]{\seq P, Γ}}
            {\seq P, Γ}
        }
        \quad \rightsquigarrow \quad
        \vcenter{
        \prfsummary[$π$]{\seq P, Γ}
        }
        $$
    \end{itemize}
\end{frame}

% \begin{frame}
%     \frametitle{Composition}
%     Reveals the role of the cut rule as a mean to \textbf{compose} proofs, in the same way that
%     application is used to compose functions.
%     \vspace{1em}
%     {
%     \fontsize{7pt}{7.2}\selectfont
%     $$
%     \vcenter{
%     \prftree[r]{\irule{\mathrm{cut}}}
%         {\prftree[r]{\irule{\parr}}
%             {\prfsummary[$π_1$]{\seq P_1, P_2, Γ}}
%             {\seq P_1 \parr P_2, Γ}}
%         {\prftree[r]{\irule{\otimes}}
%             {\prfsummary[$π_2$]{\seq P_1^\bot, Δ}}
%             {\prfsummary[$π_3$]{\seq P_2^\bot, Σ}}
%             {\seq P_1^\bot \otimes P_2^\bot, Δ, Σ}}
%         {\seq Γ, Δ, Σ}
%     }
%     \quad \rightsquigarrow \quad
%     \vcenter{
%     \prftree[r]{\irule{\mathrm{cut}}}
%         {\prftree[r]{\irule{\mathrm{cut}}}
%             {\prfsummary[$π_1$]{\seq P_1, P_2, Γ}}
%             {\prfsummary[$π_2$]{\seq P_1^\bot, Δ}}
%             {\seq P_2, Γ, Δ}}
%         {\prfsummary[$π_3$]{\seq P_2^\bot, Σ}}
%         {\seq Γ, Δ, Σ}
%     }
%     $$
%     }
%     Though the notation is not very practical for programming...
% \end{frame}

\begin{frame}
    \frametitle{Permutation of rules}
    \begin{itemize}
        \setlength\itemsep{1em}
        \item Cut elimination \textbf{terminates}
        \item But it is not \textbf{confluent}
        \item This gives rise to artificial \textbf{orderings} on rules:
        \begin{mathpar}
            \prftree[r]{\irule{\parr}}
                {\prftree[r]{\irule{\otimes}}
                    {\prfsummary[$π_1$]{\seq A, C, D, Γ}}
                    {\prfsummary[$π_2$]{\seq B, Δ}}
                    {\seq A \otimes B, C, D, Γ, Δ}}
                {\seq A \otimes B, C \parr D, Γ, Δ}
            \and
            \prftree[r]{\irule{\otimes}}
                {\prftree[r]{\irule{\parr}}
                    {\prfsummary[$π_1$]{\seq A, C, D, Γ}}
                    {\seq A, C \parr D, Γ}}
                {\prfsummary[$π_2$]{\seq B, Δ}}
                {\seq A \otimes B, C \parr D, Γ, Δ}
        \end{mathpar}
    \end{itemize}
\end{frame}

\subsection{Proof structures}

\begin{frame}
    \frametitle{Proof structures}
    \begin{itemize}
        \setlength\itemsep{1em}
        \item To \emph{equate} proofs which only differ by permutation of rules, Girard devised a
        new syntax of \textbf{proof structures}, which are \textbf{graphs} rather than
        \textbf{trees}
        \item The \textbf{desequentialization} function $\mathsf{deseq}_π$ maps sequent calculus
        proofs to their equivalent proof structures
        \item The two preceding examples collapse to:
        \vspace{1em}
        \begin{proofnet}
            \pnsomenet[R1]{\tiny$\mathsf{deseq}_π(π_1)$}{3cm}{1cm}
            \pnsomenet[R2]{\tiny$\mathsf{deseq}_π(π_2)$}{3cm}{1cm}[at (4,0)]
            \pnoutfrom{R1.-150}{Γ}
            \pnoutfrom{R1.-120}[C]{$C$}
            \pnoutfrom{R1.-60}[D]{$D$}
            \pnoutfrom{R1.-30}[A]{$A$}
            \pnoutfrom{R2.-140}[B]{$B$}
            \pnoutfrom{R2.-40}{Δ}
            \pnparr{C,D}{$C \parr D$}
            \pntensor{A,B}{$A \otimes B$}
        \end{proofnet}
    \end{itemize}
\end{frame}

\subsection{Proof nets}

% \begin{frame}
%     \frametitle{Proof nets}
%     \begin{itemize}
%         \setlength\itemsep{2em}
%         \item \textbf{Proof nets} are the image of desequentialization $\mathsf{deseq}_π$
%         \item Equivalently, they are the \textbf{sequentializable} proof structures
%         \item Then a natural question arises:
%         \vspace{1em}
%         \center
%         {\large Are \emph{all} proof structures sequentializable?}\\[2mm]
%         {\scriptsize Spoiler: \textbf{no}!}
%     \end{itemize}
% \end{frame}

\begin{frame}
    \frametitle{Proof nets}
    \begin{itemize}
        \setlength\itemsep{1em}
        \item \textbf{Proof nets} are the image of desequentialization $\mathsf{deseq}_π$
        \item Equivalently, they are the \textbf{sequentializable} proof structures
        \item \textbf{Cut elimination} on proof nets \textbf{terminates} and is \textbf{confluent}
        \item \textbf{Not all} proof structures are sequentializable:
        \vspace{-3em}
        $$
        \pnet{
            \pnformulae{
                \pnf[nP]{$P^\bot$}~~\pnf[P]{$P$}
            }
            \pnaxiom{nP,P}
            \pncut{nP,P}
        }
        $$
    \end{itemize}
    \vspace{-2em}
    \centering \textbf{Contradicts} termination of cut elimination!
\end{frame}

\subsection{Correctness}

\begin{frame}
    \frametitle{Correctness criterions}
    A way of checking the \textbf{correctness} of a proof structure without having to guess one if
    its underlying sequent calculus proofs!
    \vspace{1em}
    \begin{theorem}[Sequentialization]
        A proof structure $\mathfrak{S}$ is a proof net \emph{iff} it satisfies (all) correctness
        criterions.
    \end{theorem}
    \vspace{1em}
    \begin{block}{The \textrm{DR} criterion}
        \vspace{1em}
        \begin{itemize}
            \setlength\itemsep{1em}
            \item A \textbf{switching graph} of $\mathfrak{S}$ is $\mathfrak{S}$ where every
            $\parr$-node has one of its premisses \emph{erased}
            \item $\mathfrak{S}$ is \textbf{\textrm{DR}-correct} if all its switching graphs are
            \textbf{connected} and \textbf{acyclic}
        \end{itemize}
    \end{block}
\end{frame}

% \begin{frame}
%     % \frametitle{Counter-example $\mathfrak{S}_1$ (the vicious circle)}
%     \frametitle{Counter-example: the vicious circle}
%     % \begin{columns}
%     %     \begin{column}{0.5\textwidth}
%             $$
%             \pnet{
%                 \pnformulae{
%                     \pnf[nP]{$P^\bot$}~~\pnf[P]{$P$}
%                 }
%                 \pnaxiom{nP,P}
%                 \pncut{nP,P}
%             }
%             $$
%     %     \end{column}
%     %     \begin{column}{0.5\textwidth}
%     %         $$
%     %         \prftree[r]{\irule{\mathrm{cut}}}
%     %             {\prfbyaxiom{\irule{\mathrm{ax}}}{\seq P^\bot, P}}
%     %             {\seq}
%     %         $$
%     %     \end{column}
%     % \end{columns}
%     \begin{itemize}
%         \item No conclusion = proof of $\seq$
%         \item Impossible because of cut elimination
%     \end{itemize}
% \end{frame}

\begin{frame}[label=current]
    \frametitle{Examples}
    \vspace{-2em}
    \begin{columns}
        \begin{column}{0.5\textwidth}
            \only<3->{\color{Green}}
            \only<1>{
            $$
            \pnet{
                \pnformulae{
                    \pnf[nP1]{$P^\bot$}~~\pnf[P1]{$P$}~~\pnf[nP2]{$P^\bot$}~~\pnf[P2]{$P$}
                }
                \pnaxiom{nP1,P1}
                \pnaxiom{nP2,P2}
                \pnparr{P1,P2}{}[2.2][0.5]
                \pntensor{nP1,nP2}{}[2.2][-0.5]
            }
            $$
            }
            \only<2>{
            $$
            \pnet{
                \pnformulae{
                    \pnf[nP1]{$P^\bot$}~~\pnf[P1]{$P$}~~\pnf[nP2]{$P^\bot$}~~\pnf[P2]{$P$}
                }
                \pnaxiom{nP1,P1}
                \pnaxiom{nP2,P2}
                \pnparr{P1,P2*}{}[2.2][0.5]
                \pntensor{nP1,nP2}{}[2.2][-0.5]
            }
            $$
            }
            \only<3->{
            $$
            \pnet{
                \pnformulae{
                    \pnf[nP1]{$P^\bot$}~~\pnf[P1]{$P$}~~\pnf[nP2]{$P^\bot$}~~\pnf[P2]{$P$}
                }
                \pnaxiom{nP1,P1}
                \pnaxiom{nP2,P2}
                \pnparr{P1*,P2}{}[2.2][0.5]
                \pntensor{nP1,nP2}{}[2.2][-0.5]
            }
            $$
            }
        \end{column}
        \begin{column}{0.5\textwidth}
            \only<6->{\color{red}}
            \only<1-3>{
            $$
            \pnet{
                \pnformulae{
                    \pnf[nP1]{$P^\bot$}~~\pnf[P1]{$P$}~~\pnf[nP2]{$P^\bot$}~~\pnf[P2]{$P$}
                }
                \pnaxiom{nP1,P1}
                \pnaxiom{nP2,P2}
                \pnshpos{nP1}[shpos1]{}
                \pnshpos{nP2}[shpos2]{}
                \pnparr{shpos1,shpos2}[parr1]{}[1][-0.5]
                \pnparr{P1,P2}[parr2]{}[2.2][0.5]
                \pntensor{parr1,parr2}{}
            }
            $$
            }
            \only<4>{
            $$
            \pnet{
                \pnformulae{
                    \pnf[nP1]{$P^\bot$}~~\pnf[P1]{$P$}~~\pnf[nP2]{$P^\bot$}~~\pnf[P2]{$P$}
                }
                \pnaxiom{nP1,P1}
                \pnaxiom{nP2,P2}
                \pnshpos{nP1}[shpos1]{}
                \pnshpos{nP2}[shpos2]{}
                \pnparr{shpos1*,shpos2}[parr1]{}[1][-0.5]
                \pnparr{P1,P2*}[parr2]{}[2.2][0.5]
                \pntensor{parr1,parr2}{}
            }
            $$
            }
            \only<5>{
            $$
            \pnet{
                \pnformulae{
                    \pnf[nP1]{$P^\bot$}~~\pnf[P1]{$P$}~~\pnf[nP2]{$P^\bot$}~~\pnf[P2]{$P$}
                }
                \pnaxiom{nP1,P1}
                \pnaxiom{nP2,P2}
                \pnshpos{nP1}[shpos1]{}
                \pnshpos{nP2}[shpos2]{}
                \pnparr{shpos1,shpos2*}[parr1]{}[1][-0.5]
                \pnparr{P1*,P2}[parr2]{}[2.2][0.5]
                \pntensor{parr1,parr2}{}
            }
            $$
            }
            \only<6>{
            $$
            \pnet{
                \pnformulae{
                    \pnf[nP1]{$P^\bot$}~~\pnf[P1]{$P$}~~\pnf[nP2]{$P^\bot$}~~\pnf[P2]{$P$}
                }
                \pnaxiom{nP1,P1}
                \pnaxiom{nP2,P2}
                \pnshpos{nP1}[shpos1]{}
                \pnshpos{nP2}[shpos2]{}
                \pnparr{shpos1,shpos2*}[parr1]{}[1][-0.5]
                \pnparr{P1,P2*}[parr2]{}[2.2][0.5]
                \pntensor{parr1,parr2}{}
            }
            $$
            }
            \only<7->{
            $$
            \pnet{
                \pnformulae{
                    \pnf[nP1]{$P^\bot$}~~\pnf[P1]{$P$}~~\pnf[nP2]{$P^\bot$}~~\pnf[P2]{$P$}
                }
                \pnaxiom{nP1,P1}
                \pnaxiom{nP2,P2}
                \pnshpos{nP1}[shpos1]{}
                \pnshpos{nP2}[shpos2]{}
                \pnparr{shpos1*,shpos2}[parr1]{}[1][-0.5]
                \pnparr{P1*,P2}[parr2]{}[2.2][0.5]
                \pntensor{parr1,parr2}{}
            }
            $$
            }
        \end{column}
    \end{columns}
\end{frame}

\section{Towards ludics}

\begin{frame}[fragile, label=current]
    \frametitle{Towards ludics}
    \color{black!25}
    \begin{center}
    \adjustbox{height=0.45\textheight,keepaspectratio}{
        \begin{tikzcd}[row sep=2em, column sep=2em, math mode=false]
            & \parbox{2.5cm}{\centering System F\\\scriptsize (Girard, 1972)} \arrow[d] & & \\
            \color{black!0} \parbox{2.5cm}{\centering \bf \color{black} Proof nets\\\scriptsize (Girard, 1987)} \arrow[ddr] \arrow[ddddrr, Rightarrow, bend right] & \color{black!0} \parbox{2.5cm}{\centering \bf \color{black} Linear Logic\\\scriptsize (Girard, 1987)} \arrow[l, Rightarrow, black]\arrow[dr] \arrow[rrddd, Rightarrow, bend left, right, black] & & \\
            & & \parbox{2.5cm}{\centering Ludics\\\scriptsize (Girard, 2001)} \arrow[dl]\arrow[ddr] & \\
            & \parbox{2.5cm}{\centering L-nets\\\scriptsize(FMC, 2005)} \arrow[ddr] & & \\
            & & & \color{black!0} \parbox{2.5cm}{\centering \bf \color{black} c-designs\\\scriptsize (Terui, 2011)} \arrow[dl, Rightarrow] \\
            & & \bf This work! & \\
        \end{tikzcd}
    }
    \end{center}
\end{frame}

\subsection{The daimon $\dai$}

\begin{frame}
    \frametitle{\textrm{CP} criterion}
    \begin{itemize}
        \setlength\itemsep{1em}
        \item \textbf{Idea:} $\mathfrak{S}$ proves $A$ if it wins against every
        \textbf{counterproof}, that is every proof of $A^\bot$
        \item \textbf{Problem:} Having proofs of $A$ \emph{and} $A^\bot$ would make our logic
        \textbf{inconsistent}!
        \item Well, if we can still characterize correct proofs... Let's try a new kind of
        axiom, the \textbf{daimon} $\dai$:
        \vspace{-2em}
        \begin{mathpar}
            \prfbyaxiom{\irule{\dai}}{\seq P_1, \ldots, P_n}
            \and
            \pnet{
                \pnformulae{
                    \pnf[P1]{$P_1$}~~\pnf[ldots]{$\ldots$}~~\pnf[Pn]{$P_n$}
                }
                \pndaimon{P1,ldots,Pn}
            }
        \end{mathpar}
    \end{itemize}
\end{frame}

\subsection{Cut elimination}

\begin{frame}
    \frametitle{Cut elimination}
    \vspace{-4em}
    \begin{displaymath}
        \setcellgapes{8pt}
        \makegapedcells
        \begin{array}{r@{\qquad \rightsquigarrow \qquad}l}
            \makecell[r]{
                \pnet{
                    \pnsomenet[R]{\scriptsize$\mathfrak{S}$}{2cm}{1cm}[at (3,1)]
                    \pnformulae{
                        \pnf[Γ]{Γ}~~\pnf[P]{$P$}}
                    \pnoutfrom{R.-140}[nP]{$P^\bot$}
                    \pnoutfrom{R.-40}[Δ]{Δ}
                    \pndaimon{Γ,P}
                    \pncut{P,nP}
                }
            }
            &
            \makecell[l]{
                \pnet{
                    \pnformulae{
                        \pnf[Γ]{Γ}[at (1,1)]~~\pnf[Δ]{Δ}[at (1,1)]}
                    \pndaimon{Γ,Δ}
                }
            }
        \end{array}
    \end{displaymath}
    \vspace{-2em}
    \begin{itemize}
        \setlength\itemsep{1em}
        \item A proof that uses $\dai$ is called a \textbf{paraproof}
        % \item Paraproofs \textbf{interact} by cutting their dual conclusions
        \item $\dai$ \textbf{absorbs} $\mathfrak{S}$ instead of keeping it like $\mathrm{ax}$
        \item $\mathfrak{S} \perp \mathfrak{S}'$ if cutting their dual conclusions evaluates to
        $\dai$
    \end{itemize}
\end{frame}

\begin{frame}
    \frametitle{Correctness}
    % \begin{itemize}
    %     \item $\mathfrak{S}$ wins its interaction with $\mathfrak{S}'$ if they evaluate to $\dai$
    %     \item In that case, we say that $\mathfrak{S}$ is \textbf{orthogonal} to $\mathfrak{S}'$
    %     (written $\mathfrak{S} \perp \mathfrak{S}'$)
    % \end{itemize}
    \begin{block}{\textrm{CP} criterion}
        $\mathfrak{S}$ is \textrm{CP}-correct if for every
        \textbf{\textcolor{magenta}{counterproof}} $\mathfrak{S'}$, $\mathfrak{S} \perp
        \mathfrak{S}'$.
    \end{block}
    \begin{block}{\textrm{DR} criterion}
        $\mathfrak{S}$ is \textrm{DR}-correct if for every \textbf{\textcolor{magenta}{switching}}
        $\mathfrak{S'}$, $\mathfrak{S}'$ connected and acyclic.
    \end{block}
    \vspace{1em}
    In fact, \textrm{DR} is just a \textbf{static} reformulation of \textrm{CP}:
    \vspace{1em}
    \begin{itemize}
        \setlength\itemsep{1em}
        \item \textbf{Termination} of interaction is replaced by \textbf{acyclicity}
        \item \textbf{Uniqueness} of result (one $\dai$) is replaced by \textbf{connectedness}
    \end{itemize}

    % In fact, \textrm{DR} is just \textrm{CP} where \emph{infinite} interaction is avoided by
    % the \textbf{acyclicity} condition!\\[1em]
    % \textbf{Connectedness} rejects the \textrm{mix} rule:
    % $$\prftree[r]{\irule{\mathrm{mix}}}{\seq Γ}{\seq Δ}{\seq Γ,Δ}$$
\end{frame}

\subsection{Multiplicative c-designs}

% \begin{frame}
%     \frametitle{Ludics}
%     \begin{itemize}
%         \setlength\itemsep{1em}
%         \item Introduced by Girard (2001) as a foundational theory of logic based on
%         \textbf{interaction}
%         \item The main objects, \textbf{designs}, are idealized proofs of linear sequent calculus in
%         the same way that pure $\lambda$-terms (böhm trees) are idealized proofs of intuitionistic
%         natural deduction
%         \item Terui (2011) introduced \textbf{c-designs} as a term syntax for designs, augmented
%         with \emph{axioms} and \emph{cuts}:
%         \vspace{1em}
%         \begin{align*}
%             P &::= \dai ∣ Ω ∣ N_0 \cutbar \b{a}(\vec{N}) \\
%             N &::= x ∣ \sum_a{a(\vec{x_a}).P_a}
%         \end{align*}
%     \end{itemize}
% \end{frame}

\begin{frame}
    \frametitle{Multiplicative c-designs}
    \begin{align*}
        P ~&::=~ \dai(\vec{x}) ~∣~ Ω ~∣~ N_0 \cutbar \otimes(N_1, N_2) ~∣~ N_0 \cutbar \shpos(N_1) \\
        N ~&::=~ x ~∣~ \parr(x_1, x_2).P ~∣~ \shneg(x).P
    \end{align*}
    \begin{itemize}
        \setlength\itemsep{1em}
        % \item Restriction to the \textbf{multiplicative} fragment by removing additive
        % superimposition $\sum_a$
        % \item Abstract \textbf{actions} $a$ replaced by concrete \textbf{connectives} $\otimes,
        % \parr, \shpos, \shneg$
        % \item Added \textbf{variables} $\vec{x}$ encoding conclusions of $\dai$
        \item \textbf{Specialization} of Terui's c-designs, introduced as a \textbf{term syntax} for
        exporting \textbf{ludics} to computability/complexity theory:
        $$\mathsf{M} \text{ recognizes } \mathcal{L} \quad \Leftrightarrow \quad \mathfrak{D}_{\mathsf{M}} \perp \mathfrak{D}_{\mathcal{L}}$$
        \item Used here to encode sequent calculus paraproofs
        \item \textbf{Co-inductive} definition $\Rightarrow$ might be \textbf{infinite}!
    \end{itemize}
\end{frame}

\section{Desequentialization of terms}

\begin{frame}[fragile, label=current]
    \frametitle{Desequentialization of terms}
    \color{black!25}
    \begin{center}
    \adjustbox{height=0.45\textheight,keepaspectratio}{
        \begin{tikzcd}[row sep=2em, column sep=2em, math mode=false]
            & \parbox{2.5cm}{\centering System F\\\scriptsize (Girard, 1972)} \arrow[d] & & \\
            \color{black!0} \parbox{2.5cm}{\centering \bf \color{black} Proof nets\\\scriptsize (Girard, 1987)} \arrow[ddr] \arrow[ddddrr, Rightarrow, bend right, black] & \color{black!0} \parbox{2.5cm}{\centering \bf \color{black} Linear Logic\\\scriptsize (Girard, 1987)} \arrow[l, Rightarrow, black]\arrow[dr] \arrow[rrddd, Rightarrow, bend left, right, black] & & \\
            & & \parbox{2.5cm}{\centering Ludics\\\scriptsize (Girard, 2001)} \arrow[dl]\arrow[ddr] & \\
            & \parbox{2.5cm}{\centering L-nets\\\scriptsize(FMC, 2005)} \arrow[ddr] & & \\
            & & & \color{black!0} \parbox{2.5cm}{\centering \bf \color{black} c-designs\\\scriptsize (Terui, 2011)} \arrow[dl, Rightarrow, black] \\
            & & \bf \color{black!0} \color{black} This work! & \\
        \end{tikzcd}
    }
    \end{center}
\end{frame}

\subsection{Paranets}

\begin{frame}
    \frametitle{Paranets}
    \small
    \tabulinesep=1mm
    \begin{itemize}
        \item A term syntax for \textbf{paraproof structures}
        \item Inspired by the \textbf{differential interaction nets} of D. Mazza (2016)
        \item A \textbf{cell} is an expression of one of the following forms:\\[5pt]
        \begin{tabu}{l@{\hspace{1em}}l}
            {\normalfont \bf daimon:} $\dai(\vec{x})$, \color{violet} $\mathrm{gc}(\vec{x};\vec{y})$ &
            {\normalfont \bf multiplicative cells:} $\otimes(x; y, z)$, $\parr(x; y, z)$ \\
            {\normalfont \bf box:} \color{violet} $\mathrm{box}(\vec{x}; \vec{x}')$ &
            {\normalfont \bf shift cells:} $\shpos(x; y)$, $\shneg(x; y)$
        \end{tabu}
        \item \textbf{Ports} $x,y,z$ are \textbf{free} (resp. \textbf{bound}) when they occur
        \emph{once} (resp. \emph{twice})
        \item A \textbf{paranet} is a multiset of cells and \textbf{wires} $x \leftrightarrow y$,
        with multiset union denoted by $\mid$
    \end{itemize}
    \begin{center}
    \begin{tabu}{>{\bf\color{blue}}c@{$\quad \Longleftrightarrow \quad$}>{\bf\color{red}}c}
        cell & node \\
        bound port/wire & edge \\
        free port & conclusion
    \end{tabu}
    \end{center}
\end{frame}

\subsection{Desequentialization}

\begin{frame}[fragile]
    \frametitle{Desequentialization}
    This is a \textbf{two-step} procedure:
    \vspace{1em}
    \begin{mathpar}
        \begin{tikzcd}[row sep=large, column sep=large]
            \mathcal{D} \arrow[r, "\mathsf{B}"] & \mathcal{B} \arrow[r, "\mathsf{deseq}_μ"] & \mathcal{N} \\
        \end{tikzcd}
    \end{mathpar}
    \vspace{-2em}
    \begin{enumerate}
        \setlength\itemsep{1em}
        \item \textbf{Traduction} $\mathsf{B}$ from multiplicative c-designs to paranets
        \item \textbf{Removal of boxes} through rewriting steps $\mathsf{deseq}_μ$. Two variants:
        \begin{itemize}
            \item $\mathsf{deseq}_μ^n$ ("big-step"): remove entire boxes
            \item $\mathsf{deseq}_μ^1$ ("small-step"): remove wire by wire
        \end{itemize}
    \end{enumerate}
\end{frame}

\subsection{Correction}

\begin{frame}[fragile]
    \frametitle{Correction}
    \small
    \vspace{-2em}
    \begin{mathpar}
        \begin{tikzcd}[row sep=large, column sep=large]
            \mathsf{MLL}\dai \arrow[rr, "\mathsf{deseq}_\pi"]\arrow[d, "\mathsf{D}"] & & \mathsf{PN}_{\mathsf{MLL}\dai} \arrow[d, "\mathsf{N}"] \\
            \mathcal{D} \arrow[r, "\mathsf{B}"]\arrow[d, "\mathsf{\cutred_{\mathcal{D}}}"] & \mathcal{B} \arrow[r, "\mathsf{deseq}_μ"]\arrow[d, "\mathsf{\cutred_{\mathcal{B}}}"] & \mathcal{N} \arrow[d, "\mathsf{\cutred_{\mathcal{B}}}"] \\
            \mathcal{D} \arrow[r, "\mathsf{B}"] & \mathcal{B} \arrow[r, "\mathsf{deseq}_μ"] & \mathcal{N} \\
        \end{tikzcd}
    \end{mathpar}
    \vspace{-2em}
    \begin{itemize}
        \item \textbf{Static} correction (top): desequentialization of terms simulates desequentialization of proofs
        \item \textbf{Dynamic} correction (bottom): desequentialization of terms commutes with cut elimination on terms
    \end{itemize}
\end{frame}

\section*{Conclusion}

\begin{frame}
    \frametitle{Conclusion}
    \begin{block}{What we have done}
        \begin{itemize}
        \item Introduced and related sequential, parallel and interactive proof systems for
        multiplicative linear logic
        \item Designed and introduced term syntaxes for those systems
        \item Related the sequential and parallel syntaxes through desequentialization
        \end{itemize}
    \end{block}
    \begin{block}{Future work}
        \begin{itemize}
        \item Proving that our desequentialization is correct
        \item Importing results such as correctness criterions in our syntax
        \item Extending our syntax to the additive fragment of linear logic, and abstracting
        away from connectives: this would lead us to L-nets
        \end{itemize}
    \end{block}
\end{frame}

\begin{frame}
    \frametitle{Bibliography}
    \bibliographystyle{amsalpha}
    \bibliography{main.bib}
\end{frame}

\end{document}